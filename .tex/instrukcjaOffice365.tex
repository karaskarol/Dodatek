\documentclass[a4paper,12pt]{article}
\usepackage[utf8]{inputenc}
\usepackage[T1]{fontenc}
\usepackage[polish]{babel}
\usepackage[a4paper,margin=2.5cm]{geometry}
\usepackage{hyperref}
\usepackage{tocloft}
\usepackage{listings}
\usepackage{xcolor}

% Konfiguracja wyglądu spisu treści
\setlength{\cftbeforesecskip}{4pt}
\renewcommand{\cftsecfont}{\bfseries}
\renewcommand{\cftsecpagefont}{\bfseries}

% Konfiguracja środowiska listings
\lstset{
  basicstyle=\ttfamily\small,
  keywordstyle=\color{blue},
  commentstyle=\color{gray},
  breaklines=true,
  frame=single,
  columns=flexible
}

\title{Przewodnik Instalacji Office~365 z Office Deployment Tool}
\author{}
\date{\today}

\begin{document}

\maketitle

\begin{abstract}
Ten dokument zawiera skondensowany, ale kompletny opis kroków potrzebnych do pobrania i instalacji Office~365 przy użyciu Office Deployment Tool (ODT). Każdy etap obejmuje niezbędne informacje i praktyczne wskazówki.
\end{abstract}

\tableofcontents
\newpage

\section{Wymagania wstępne}
Przed instalacją upewnij się, że:
\begin{itemize}
  \item Posiadasz uprawnienia administratora na komputerze.
  \item Masz stabilne połączenie internetowe.
  \item Dysk systemowy ma co najmniej 4~GB wolnego miejsca.
\end{itemize}

\section{Pobranie Office Deployment Tool}
Office Deployment Tool (ODT) to oficjalne narzędzie Microsoft do niestandardowych instalacji Office.
\begin{enumerate}
  \item Przejdź na stronę: \url{https://www.microsoft.com/software-download/office}.
  \item Pobierz plik \texttt{setup.exe} (aktualna wersja ODT).
  \item Uruchom go, wybierając folder docelowy (np. \texttt{C:/OfficeSetup}).
\end{enumerate}

\section{Przygotowanie pliku konfiguracyjnego XML}
Plik \texttt{Configuration.xml} steruje zawartością instalacji.
\subsection{Struktura pliku}
\begin{lstlisting}[language=XML]
<Configuration>
  <Add OfficeClientEdition="64" Channel="Current">
    <Product ID="O365ProPlusRetail">
      <Language ID="pl-pl"/>
      <ExcludeApp ID="Access"/>
      <ExcludeApp ID="Publisher"/>
    </Product>
  </Add>
  <Display Level="Full" AcceptEULA="TRUE"/>
  <Property Name="AUTOACTIVATE" Value="1"/>
</Configuration>
\end{lstlisting}

\subsection{Opis elementów}
\begin{description}
  \item[\texttt{OfficeClientEdition}] Wersja 64-bitowa (lub 32).
  \item[\texttt{Channel}] Gałąź aktualizacji (\texttt{Current}, \texttt{Monthly}, \dots).
  \item[\texttt{Language}] Język instalacji (tutaj polski).
  \item[\texttt{ExcludeApp}] Wyłączenie aplikacji Access i Publisher.
  \item[\texttt{Display}] Poziom widoczności instalatora i akceptacja EULA.
  \item[\texttt{Property AUTOACTIVATE}] Automatyczna aktywacja produktu.
\end{description}

\section{Instalacja Office~365}
\begin{enumerate}
  \item Otwórz \textbf{Wiersz polecenia} jako administrator.
  \item Przejdź do katalogu z narzędziem:
    \begin{lstlisting}
cd C:/OfficeSetup
    \end{lstlisting}
  \item Uruchom polecenie instalacji:
    \begin{lstlisting}
setup.exe /configure Configuration.xml
    \end{lstlisting}
  \item Poczekaj na zakończenie procesu — pasek postępu wskaże status.
\end{enumerate}

\section{Weryfikacja poprawności}
Po zakończeniu:
\begin{itemize}
  \item Uruchom np. Microsoft Word.
  \item W menu \texttt{Plik > Konto} sprawdź wersję (64-bit) i język interfejsu.
\end{itemize}

\section{Rozwiązywanie problemów}
Jeśli instalacja się nie powiedzie:
\begin{itemize}
  \item Zweryfikuj \texttt{Configuration.xml} pod kątem składni.
  \item Sprawdź połączenie internetowe.
  \item Przejrzyj logi w katalogu \texttt{C:/OfficeSetup} (pliki \texttt{*.log}).
  \item Skorzystaj z dokumentacji: \url{https://docs.microsoft.com/officedeployment}
\end{itemize}

\end{document}
